\documentclass[12pt]{article}

% Packages
\usepackage[margin=1in]{geometry}
\usepackage{setspace}
\usepackage{graphicx}
\usepackage{booktabs}
\usepackage{amsmath}
\usepackage{amssymb}
\usepackage{natbib}
\usepackage{hyperref}
\usepackage{float}
\usepackage{caption}
\usepackage{subcaption}
\usepackage{threeparttable}
\usepackage{rotating}
\usepackage{pdflscape}
\usepackage{array}
\usepackage{multirow}
\usepackage{longtable}
\usepackage{appendix}

% Settings
\doublespacing
\hypersetup{
    colorlinks=true,
    linkcolor=blue,
    citecolor=blue,
    urlcolor=blue
}

% Title
\title{\textbf{Import Substitution and Labor Markets: Evidence from Russia's Food Embargo, 2014--2023}}

\author{
    [Author Name]\\
    \textit{Florida Institute of Technology}\\
    \texttt{email@fit.edu}
}

\date{\today\\[1em]\textit{Preliminary Draft -- Please Do Not Cite}}

\begin{document}

\maketitle

\begin{abstract}
\noindent We study how domestic labor markets respond to sudden trade barriers using Russia's 2014 food import embargo as a natural experiment. Using individual-level panel data from the Russia Longitudinal Monitoring Survey (RLMS), we employ a difference-in-differences design comparing agricultural workers to those in other sectors. \textbf{Our primary sample covers 2010--2019}, providing a clean 5-year pre/post window uncontaminated by later shocks (COVID-19, Ukraine war). Using intent-to-treat specifications, we find agricultural workers experienced earnings gains of 8--9 percent. Our wage decomposition suggests these gains partly reflect differences in hours worked, though pre-existing hours differences between sectors complicate causal interpretation. Strikingly, agricultural employment \textit{declined} 36\% despite these earnings gains, suggesting workers value non-wage amenities over the protection premium. We present 2020--2023 results separately with explicit caveats about confounding from pandemic and war disruptions. Our findings suggest trade protection generates earnings gains through increased labor demand, but workers respond on the hours margin rather than through sectoral reallocation.

\bigskip
\noindent \textbf{JEL Codes:} F13, F14, J22, J31, Q17

\bigskip
\noindent \textbf{Keywords:} Trade protection, import substitution, wages, hours worked, agriculture, Russia, sanctions
\end{abstract}

\newpage
\tableofcontents
\newpage

%==============================================================================
\section{Introduction}
%==============================================================================

How do domestic labor markets respond to sudden trade barriers? While a large literature examines the labor market effects of trade liberalization \citep{autor2013china, dix2017trade, pierce2016surprisingly}, less is known about the effects of trade protection, particularly in the long run. This paper exploits the natural experiment of Russia's 2014 food import embargo to study the dynamic effects of trade barriers on wages in protected industries.

In August 2014, Russia imposed a ban on food imports from the United States, European Union, Canada, Australia, and Norway in response to Western sanctions over the Ukraine crisis. The embargo covered meat, dairy products, fish, fruits, and vegetables---products for which Russia had significant import dependence (ranging from 15\% to 65\% of domestic consumption). Unlike gradual tariff changes, this policy was sudden, unexpected, and comprehensive, providing a clean identification strategy for studying the effects of trade protection.

Our setting offers three key advantages for identification. First, the embargo was an exogenous shock driven by geopolitical events, not by domestic economic conditions or lobbying by agricultural interests. Second, the policy affected specific product categories, allowing us to compare workers in affected industries to those in unaffected sectors. Third, the embargo has remained in place for over a decade and has been repeatedly extended, allowing us to trace labor market adjustments over an unusually long time horizon.

We use individual-level panel data from the Russia Longitudinal Monitoring Survey (RLMS), which tracks the same individuals over time from 2010 to 2023. This allows us to include individual fixed effects, controlling for time-invariant worker characteristics, and to follow workers' wage trajectories before and after the policy change. We complement this with firm-level data from the Russian Firm Statistical Database (RFSD) to construct regional measures of exposure to the embargo based on the pre-existing agricultural composition of each region.

Our main finding is that agricultural workers experienced relative wage gains following the embargo. In our baseline difference-in-differences specification with individual fixed effects, we estimate that wages in agriculture increased by approximately 3.6 percent relative to other sectors after 2014. When we add demographic controls, this estimate rises to 4.6 percent and becomes statistically significant at the 5 percent level.

Perhaps more importantly, our event study analysis reveals that these effects grew stronger over time rather than fading. While initial wage effects in 2014--2016 were small and imprecisely estimated, by 2020--2023 (six to nine years after the embargo), agricultural workers' wages had increased by 8--10 percent relative to the pre-treatment period. This pattern of growing effects is consistent with gradual import substitution and capacity building in domestic agriculture.

Our paper contributes to several literatures. First, we contribute to the literature on trade and labor markets by providing evidence on the effects of trade protection, complementing the extensive work on trade liberalization \citep{autor2013china, kovak2013regional, topalova2010trade}. Second, we contribute to the literature on import substitution by documenting labor market effects of this policy approach \citep{bruton1998reconsideration}. Third, we provide evidence on the long-run persistence of trade policy effects, which has been difficult to study due to data limitations and the typically gradual nature of trade policy changes.

The remainder of this paper proceeds as follows. Section \ref{sec:background} provides background on Russia's food embargo. Section \ref{sec:data} describes our data sources. Section \ref{sec:empirical} presents our empirical strategy. Section \ref{sec:results} reports our main results. Section \ref{sec:robustness} presents robustness checks. Section \ref{sec:conclusion} concludes.

%==============================================================================
\section{Background: Russia's Food Import Embargo} \label{sec:background}
%==============================================================================

\subsection{The 2014 Food Ban}

On August 6, 2014, Russia announced a ban on imports of certain agricultural products from countries that had imposed sanctions on Russia over its involvement in the Ukraine crisis. The ban initially covered the United States, European Union member states, Canada, Australia, and Norway. The banned products included:

\begin{itemize}
    \item Meat (beef, pork, poultry)
    \item Fish and seafood
    \item Dairy products and cheese
    \item Fruits and vegetables
    \item Nuts
\end{itemize}

Table \ref{tab:import_shares} shows the pre-ban import shares for key product categories. Import dependence varied substantially across products, from approximately 15\% for poultry to 60--70\% for fruits and vegetables. This variation provides the basis for our heterogeneity analysis by product type.

\begin{table}[H]
\centering
\caption{Pre-Ban Import Shares by Product Category}
\label{tab:import_shares}
\begin{threeparttable}
\begin{tabular}{lc}
\toprule
Product Category & Import Share (\%) \\
\midrule
Fruits and vegetables & 60--70 \\
Dairy and cheese & 30--40 \\
Fish and seafood & 30 \\
Beef & 25 \\
Pork & 25 \\
Poultry & 15 \\
\bottomrule
\end{tabular}
\begin{tablenotes}
\small
\item \textit{Notes:} Import shares represent the share of domestic consumption supplied by imports from all countries prior to the 2014 embargo. Sources: Rosstat, UN Comtrade.
\end{tablenotes}
\end{threeparttable}
\end{table}

\subsection{Policy Extensions and Modifications}

The embargo was initially announced for one year but has been repeatedly extended. Table \ref{tab:policy_timeline} summarizes key policy changes:

\begin{table}[H]
\centering
\caption{Timeline of Food Embargo Policy Changes}
\label{tab:policy_timeline}
\begin{threeparttable}
\begin{tabular}{ll}
\toprule
Date & Policy Change \\
\midrule
August 2014 & Initial ban (US, EU, Canada, Australia, Norway) \\
August 2015 & Albania, Montenegro, Iceland, Liechtenstein added \\
January 2016 & Ukraine added \\
May 2016 & Some baby food products exempted \\
October 2017 & Live pigs and animal offal added \\
December 2020 & United Kingdom added (post-Brexit) \\
2015--2025 & Annual extensions \\
\bottomrule
\end{tabular}
\begin{tablenotes}
\small
\item \textit{Notes:} The embargo has been extended annually and is currently set to remain in effect through at least 2025.
\end{tablenotes}
\end{threeparttable}
\end{table}

\subsection{Import Substitution Outcomes}

The embargo was explicitly designed to promote domestic agricultural production through import substitution. The results have been mixed across product categories. Domestic production increased substantially for pork, poultry, and greenhouse vegetables, where Russia achieved near self-sufficiency by 2020. However, import substitution was less successful for dairy products, particularly cheese, where quality and variety remained below pre-ban import levels.

These differential outcomes across products motivate our analysis of heterogeneous effects by agricultural sub-sector.

\subsection{Theoretical Framework: A Two-Period Model}

To structure our empirical analysis and generate testable predictions about the dynamics of labor market adjustment, we present a simple two-period model of agricultural production under trade protection.

\paragraph{Setup.} Consider a representative agricultural firm producing output $Y$ using capital $K$ and labor $L$ according to:
\begin{equation}
    Y = A K^\alpha L^{1-\alpha}, \quad \alpha \in (0,1)
\end{equation}
where $A$ is productivity. Prior to the embargo, the domestic market price is determined by import competition at $p_m$. After the embargo, imports are banned and the domestic price rises to $p^* > p_m$.

\paragraph{Period 1: Short-run adjustment.} In the short run, capital is fixed at $K_0$. The firm chooses labor to maximize profits:
\begin{equation}
    \max_L \; \pi_1 = p^* A K_0^\alpha L^{1-\alpha} - w L
\end{equation}
The first-order condition yields labor demand:
\begin{equation}
    L_1^d = \left( \frac{(1-\alpha) p^* A K_0^\alpha}{w} \right)^{1/\alpha}
\end{equation}
The price increase from $p_m$ to $p^*$ shifts labor demand outward, increasing equilibrium hours. If labor supply is relatively elastic (workers willing to supply additional hours at the prevailing wage), this manifests as increased hours rather than higher wages.

\paragraph{Period 2: Capacity expansion.} In period 2, firms can invest in additional capacity. Let $I$ denote investment, with capital evolving as $K_1 = K_0 + I$. Investment faces convex adjustment costs:
\begin{equation}
    C(I) = I + \frac{\gamma}{2} I^2
\end{equation}
where $\gamma > 0$ captures installation costs, supply chain frictions, and time-to-build.

Crucially, firms face a \textit{credit constraint}: investment cannot exceed a fraction $\theta$ of period-1 profits:
\begin{equation}
    I \leq \theta \cdot \pi_1
\end{equation}
This constraint binds when firms cannot access external financing and must self-finance expansion from retained earnings.

The firm's period-2 problem is:
\begin{equation}
    \max_{L_2, I} \; \pi_2 = p^* A (K_0 + I)^\alpha L_2^{1-\alpha} - w L_2 - C(I) \quad \text{s.t.} \quad I \leq \theta \pi_1
\end{equation}

When the credit constraint binds, investment is:
\begin{equation}
    I^* = \theta \pi_1 = \theta \left[ p^* A K_0^\alpha (L_1^*)^{1-\alpha} - w L_1^* \right]
\end{equation}
Higher period-1 profits relax the credit constraint, enabling more investment, which further increases labor demand in period 2.

\paragraph{Predictions.} The model generates three testable predictions:

\begin{enumerate}
    \item \textbf{Growing effects}: Labor demand increases in both periods---first from the price shock at fixed capital, then from capacity expansion. Effects should grow over time as investment accumulates.

    \item \textbf{Hours vs. wages}: If agricultural labor supply is elastic (rural workers can increase hours, or underemployed workers enter from subsistence), the demand expansion manifests primarily in hours, not wage rates. This is consistent with a ``Lewis-type'' surplus labor model.

    \item \textbf{Persistence}: Effects persist because:
    \begin{itemize}
        \item \textit{Credit constraints} limit new firm entry, protecting incumbents
        \item \textit{Sector-specific capital} (land, equipment, expertise) creates barriers
        \item \textit{Geographic constraints}: Agriculture is predominantly rural; urban workers face high relocation costs
        \item \textit{Firm consolidation}: Large incumbents capture scale economies, crowding out potential entrants
    \end{itemize}
\end{enumerate}

\paragraph{Why doesn't labor entry erode rents?} In a frictionless model, higher labor demand would attract workers from other sectors until wages equalize. Several frictions prevent this:

\begin{itemize}
    \item \textbf{Sector-specific human capital}: Agricultural skills (equipment operation, animal husbandry, seasonal timing) are not easily acquired by urban workers.
    \item \textbf{Geographic mismatch}: Agricultural jobs are in rural areas while unemployed workers are concentrated in cities. Relocation costs are substantial.
    \item \textbf{Hours margin vs. employment margin}: Our evidence suggests adjustment occurs through \textit{incumbent workers increasing hours}, not new workers entering the sector. Agricultural employment actually \textit{declined} post-embargo.
\end{itemize}

This last point is crucial: workers are not capturing rents through higher wages because they are not scarce. Instead, existing agricultural workers are supplying more hours in response to increased labor demand, consistent with elastic labor supply at the intensive margin.

%==============================================================================
\section{Data} \label{sec:data}
%==============================================================================

\subsection{Russia Longitudinal Monitoring Survey (RLMS)}

Our primary data source is the Russia Longitudinal Monitoring Survey (RLMS-HSE), a nationally representative panel survey that has tracked Russian households and individuals since 1994. We use waves covering 2010--2023, providing four years of pre-treatment data and nine years of post-treatment data.

The RLMS contains detailed information on:
\begin{itemize}
    \item Individual labor market outcomes (wages, employment, hours worked)
    \item Industry of employment (using Russian classification codes)
    \item Demographic characteristics (age, gender, education, marital status)
    \item Geographic location (region/PSU codes)
    \item Household characteristics
\end{itemize}

Our key outcome variable is monthly after-tax wages (\texttt{j10} in the RLMS codebook). We restrict our sample to working-age individuals (18--65) who are currently employed and report positive wages. Our treatment variable is based on industry of employment: workers in agriculture (industry code 8) are classified as treated, while workers in other industries serve as the control group.

Table \ref{tab:summary_stats} presents summary statistics for our analysis sample.

\begin{table}[H]
\centering
\caption{Summary Statistics (2013 Baseline)}
\label{tab:summary_stats}
\begin{threeparttable}
\begin{tabular}{lccc}
\toprule
& All Workers & Agriculture & Other Sectors \\
\midrule
Log monthly wage & 9.48 & 9.02 & 9.51 \\
& (0.72) & (0.68) & (0.71) \\
Monthly wage (rubles) & 18,542 & 11,847 & 19,123 \\
& (15,821) & (9,426) & (16,012) \\
Hours worked (monthly) & 176 & 184 & 175 \\
& (42) & (48) & (41) \\
Age & 40.2 & 43.1 & 40.0 \\
& (11.8) & (11.2) & (11.8) \\
Female (\%) & 52.3 & 38.5 & 53.2 \\
University education (\%) & 28.4 & 8.2 & 29.8 \\
\midrule
Observations & 7,842 & 412 & 7,430 \\
Share of sample (\%) & 100 & 5.3 & 94.7 \\
\bottomrule
\end{tabular}
\begin{tablenotes}
\small
\item \textit{Notes:} Standard deviations in parentheses. Sample restricted to employed workers aged 18--65 with non-missing wages in 2013. Wages are in nominal rubles.
\end{tablenotes}
\end{threeparttable}
\end{table}

\subsection{Russian Firm Statistical Database (RFSD)}

We supplement the RLMS with firm-level data from the Russian Firm Statistical Database (RFSD), which contains balance sheet and income statement information for the universe of Russian firms from 2011--2024. We use this data to construct regional measures of agricultural intensity and treatment exposure.

Specifically, we compute for each region:
\begin{itemize}
    \item The share of firms in agriculture and food processing
    \item The product composition of agricultural firms (dairy, meat, fruits/vegetables, fish)
    \item A treatment intensity measure that weights regional agricultural composition by product-level import shares
\end{itemize}

Our treatment intensity measure for region $r$ is:
\begin{equation}
    \text{Intensity}_r = \sum_p \text{Share}_{rp} \times \text{ImportShare}_p
\end{equation}
where $\text{Share}_{rp}$ is the share of region $r$'s agricultural firms in product category $p$, and $\text{ImportShare}_p$ is the pre-ban import share for product $p$.

\subsection{Sample Size and Statistical Power}

Table \ref{tab:sample_sizes} reports effective sample sizes for our analysis. While agricultural workers represent a relatively small share of the RLMS sample (3--5\% annually), the panel structure and 10-year time span generate sufficient power for our main analysis.

\begin{table}[H]
\centering
\caption{Effective Sample Sizes}
\label{tab:sample_sizes}
\begin{threeparttable}
\begin{tabular}{lcc}
\toprule
& Agriculture & Other Sectors \\
\midrule
\multicolumn{3}{l}{\textit{Primary Sample (2010--2019)}} \\
\quad Worker-years & 3,204 & 74,891 \\
\quad Unique individuals & 766 & 14,847 \\
\quad Baseline (2013) & 368 & 8,323 \\
\quad Post-treatment (2014--2019) & 1,571 & 41,463 \\
\midrule
\multicolumn{3}{l}{\textit{Extended Sample (2010--2023)}} \\
\quad Worker-years & 4,139 & 101,003 \\
\quad Unique individuals & 892 & 16,215 \\
\midrule
\multicolumn{3}{l}{\textit{By Year (Agriculture)}} \\
\quad 2010 & 430 & \\
\quad 2013 (baseline) & 368 & \\
\quad 2014 & 273 & \\
\quad 2019 & 244 & \\
\quad 2023 & 236 & \\
\bottomrule
\end{tabular}
\begin{tablenotes}
\small
\item \textit{Notes:} Sample restricted to employed workers aged 18--65 with non-missing wages.
\end{tablenotes}
\end{threeparttable}
\end{table}

\paragraph{Power calculations.} With 3,204 agricultural worker-years in our primary sample and a within-group standard deviation of log wages of approximately 0.65, our minimum detectable effect (MDE) at 80\% power and $\alpha = 0.05$ is approximately 3.2 percentage points. Accounting for clustering at the regional level (35 clusters, design effect $\approx$ 1.5), the MDE increases to approximately 4.0 percentage points. Our estimated effect of 3.6\% is at the margin of detectability, which explains the borderline statistical significance in some specifications.

\paragraph{Sub-sector analysis: Data limitations.} The RLMS industry classification does not distinguish between agricultural sub-sectors (e.g., livestock, dairy, crops). The only available disaggregation is through ISCO occupation codes, which separate skilled agricultural workers (ISCO 6) from agricultural laborers (ISCO 92). However, even this crude classification yields cell sizes of 50--200 worker-years per category, implying MDEs of 15--25\%. We therefore \textit{do not} pursue sub-sector heterogeneity analysis, as we lack the statistical power to detect meaningful differences. Future research with larger agricultural samples or sub-sector identifiers could address this limitation.

%==============================================================================
\section{Empirical Strategy} \label{sec:empirical}
%==============================================================================

\subsection{Difference-in-Differences}

Our baseline specification is a difference-in-differences design comparing agricultural workers (treated) to workers in other sectors (control) before and after the 2014 embargo:

\begin{equation}
    \ln(W_{it}) = \alpha_i + \gamma_t + \beta (\text{Agri}_i \times \text{Post}_t) + X_{it}'\delta + \varepsilon_{it}
\end{equation}

where $\ln(W_{it})$ is the log monthly wage of individual $i$ in year $t$, $\alpha_i$ are individual fixed effects, $\gamma_t$ are year fixed effects, $\text{Agri}_i$ is an indicator for working in agriculture, $\text{Post}_t$ is an indicator for years 2014 and later, and $X_{it}$ is a vector of time-varying controls (age, age squared, education). The coefficient of interest is $\beta$, which captures the differential change in wages for agricultural workers relative to other workers after the embargo.

Standard errors are clustered at the region level to account for potential correlation of shocks within regions.

\subsection{Event Study}

To examine the dynamics of the treatment effect and assess the parallel trends assumption, we estimate an event study specification:

\begin{equation}
    \ln(W_{it}) = \alpha_i + \gamma_t + \sum_{k \neq -1} \beta_k (\text{Agri}_i \times \mathbf{1}[t - 2014 = k]) + \varepsilon_{it}
\end{equation}

where $k$ indexes years relative to the treatment year (2014). The coefficients $\beta_k$ trace out the year-by-year difference in wages between agricultural and non-agricultural workers, relative to the omitted year ($k = -1$, i.e., 2013). Under the parallel trends assumption, we expect $\beta_k \approx 0$ for $k < 0$.

\subsection{Regional Treatment Intensity}

We also exploit regional variation in exposure to the embargo using a continuous treatment intensity measure:

\begin{equation}
    \ln(W_{it}) = \alpha_i + \gamma_t + \beta (\text{Intensity}_r \times \text{Post}_t) + \varepsilon_{it}
\end{equation}

This specification tests whether workers in regions with greater agricultural intensity (and thus greater exposure to the embargo's import substitution effects) experienced larger wage gains.

\subsection{Identification Assumptions}

Our identification relies on the following assumptions:

\begin{enumerate}
    \item \textbf{Parallel trends}: In the absence of the embargo, wages in agriculture would have evolved similarly to wages in other sectors.
    \item \textbf{No anticipation}: Workers did not adjust their behavior in anticipation of the embargo (supported by the sudden, unexpected nature of the policy).
    \item \textbf{SUTVA}: The treatment status of one worker does not affect outcomes for other workers (potentially violated if there are general equilibrium effects).
\end{enumerate}

We assess the parallel trends assumption through our event study analysis by examining whether pre-treatment coefficients are close to zero.

%==============================================================================
\section{Results} \label{sec:results}
%==============================================================================

\subsection{Baseline Difference-in-Differences}

Table \ref{tab:baseline_did} presents our baseline difference-in-differences estimates.

\begin{table}[H]
\centering
\caption{Effect of Food Embargo on Agricultural Wages}
\label{tab:baseline_did}
\begin{threeparttable}
\begin{tabular}{lcccc}
\toprule
& (1) & (2) & (3) & (4) \\
& OLS & Ind. FE & Ind. FE + Controls & Agri + Food \\
\midrule
Agriculture $\times$ Post & 0.181*** & 0.036 & 0.046** & \\
& (0.044) & (0.023) & (0.022) & \\
Treated Sector $\times$ Post & & & & 0.023** \\
& & & & (0.010) \\
Agriculture & $-$0.460*** & & & \\
& (0.071) & & & \\
\midrule
Individual FE & No & Yes & Yes & Yes \\
Year FE & Yes & Yes & Yes & Yes \\
Controls & No & No & Yes & No \\
\midrule
Observations & 105,142 & 99,536 & 99,399 & 99,536 \\
R-squared & 0.18 & 0.72 & 0.73 & 0.72 \\
\bottomrule
\end{tabular}
\begin{tablenotes}
\small
\item \textit{Notes:} Dependent variable is log monthly wage. Controls include age, age squared, and education category dummies. ``Treated Sector'' includes both agriculture (industry code 8) and food/light industry (industry code 1). Standard errors clustered at region level in parentheses. * p$<$0.10, ** p$<$0.05, *** p$<$0.01.
\end{tablenotes}
\end{threeparttable}
\end{table}

Column (1) shows the simple OLS estimate without individual fixed effects. The coefficient of 0.181 suggests an 18.1\% wage increase for agricultural workers post-embargo, but this estimate is likely biased by selection into agriculture.

Column (2) adds individual fixed effects, exploiting within-person variation in wages over time. The coefficient drops to 0.036 (3.6\%), which is not statistically significant at conventional levels. This suggests that the large OLS estimate was driven by composition effects rather than causal wage gains.

Column (3) adds time-varying controls (age, age squared, education). The coefficient increases slightly to 0.046 (4.6\%) and becomes statistically significant at the 5\% level.

Column (4) expands the treated group to include both agriculture and food processing industries. The coefficient of 0.023 (2.3\%) is smaller but precisely estimated.

\subsection{Event Study}

Figure \ref{fig:event_study} presents our event study estimates, plotting the year-by-year difference in wages between agricultural and non-agricultural workers relative to 2013.

\begin{figure}[H]
\centering
\includegraphics[width=0.9\textwidth]{output/figures/event_study_main.png}
\caption{Event Study: Effect of Food Embargo on Agricultural Wages}
\label{fig:event_study}
\begin{minipage}{0.9\textwidth}
\small
\textit{Notes:} Figure plots coefficients from equation (2), showing the difference in log wages between agricultural and non-agricultural workers relative to 2013 (t = $-$1). Vertical bars show 95\% confidence intervals based on standard errors clustered at the region level. The dashed vertical line indicates the timing of the embargo (August 2014).
\end{minipage}
\end{figure}

The event study reveals several important patterns:

\begin{enumerate}
    \item \textbf{Pre-trends}: Coefficients for 2010--2012 (t = $-$4 to $-$2) are close to zero and not statistically different from the baseline, supporting the parallel trends assumption.

    \item \textbf{Initial effects}: The immediate effects in 2014--2016 (t = 0 to 2) are small and imprecisely estimated.

    \item \textbf{Growing effects}: Effects strengthen over time, reaching approximately 0.05--0.10 log points (5--10\%) by 2020--2023 (t = 6 to 9).

    \item \textbf{Persistence}: There is no evidence that effects fade over the nine-year post-treatment period.
\end{enumerate}

This pattern of growing effects is consistent with gradual import substitution: as domestic production expands and firms invest in capacity, demand for agricultural labor increases, pushing up wages.

\subsection{Wage Trends}

Figure \ref{fig:wage_trends} shows the raw wage trends for agricultural and non-agricultural workers over the sample period.

\begin{figure}[H]
\centering
\includegraphics[width=0.9\textwidth]{output/figures/wage_trends.png}
\caption{Wage Trends: Agriculture vs. Other Sectors}
\label{fig:wage_trends}
\begin{minipage}{0.9\textwidth}
\small
\textit{Notes:} Figure shows mean log monthly wages by year for agricultural workers (blue) and workers in other sectors (gray). The dashed vertical line indicates the timing of the embargo (August 2014).
\end{minipage}
\end{figure}

The figure shows that while agricultural wages remain below wages in other sectors throughout the period, the gap narrows after 2014. Before the embargo, the log wage gap was approximately 0.5 (about 50\% lower wages in agriculture). By 2023, this gap had narrowed to approximately 0.15--0.20.

\subsection{Regional Treatment Intensity}

Table \ref{tab:regional} presents results using regional variation in treatment intensity.

\begin{table}[H]
\centering
\caption{Regional Treatment Intensity}
\label{tab:regional}
\begin{threeparttable}
\begin{tabular}{lccc}
\toprule
& (1) & (2) & (3) \\
& Continuous & High/Low & Triple DiD \\
\midrule
Intensity $\times$ Post & 0.142** & & \\
& (0.068) & & \\
High Treatment $\times$ Post & & 0.031** & \\
& & (0.014) & \\
Agri $\times$ High $\times$ Post & & & 0.048* \\
& & & (0.026) \\
\midrule
Individual FE & Yes & Yes & Yes \\
Year FE & Yes & Yes & Yes \\
\midrule
Observations & 99,536 & 99,536 & 99,536 \\
\bottomrule
\end{tabular}
\begin{tablenotes}
\small
\item \textit{Notes:} ``Intensity'' is the product-weighted agricultural intensity measure. ``High Treatment'' is an indicator for above-median treatment intensity. Standard errors clustered at region level. * p$<$0.10, ** p$<$0.05, *** p$<$0.01.
\end{tablenotes}
\end{threeparttable}
\end{table}

Column (1) shows that a one-unit increase in treatment intensity is associated with a 14.2\% wage increase post-embargo. Column (2) shows that workers in high-treatment regions experienced 3.1\% higher wage growth than those in low-treatment regions. Column (3) presents a triple-difference specification, showing that agricultural workers in high-treatment regions experienced the largest gains.

\subsection{Geographic Heterogeneity: Regional Agricultural Intensity}

We further exploit geographic variation using regional agricultural employment shares computed directly from the RLMS. This approach has the advantage of using the same data source for both treatment assignment and outcomes. We compute each region's agricultural employment share in 2013 (pre-treatment) and test whether effects are larger in regions with greater agricultural intensity---a natural proxy for exposure to import substitution effects.

Table \ref{tab:geographic} presents the results. Regional agricultural shares vary substantially: the median region has 1.1\% agricultural employment, while the top regions (Amur, Penza, Volgograd oblasts) have shares of 20--38\%.

\begin{table}[H]
\centering
\caption{Geographic Heterogeneity: Effects by Regional Agricultural Intensity}
\label{tab:geographic}
\begin{threeparttable}
\begin{tabular}{lcccc}
\toprule
& (1) & (2) & (3) & (4) \\
& Overall & Low Tercile & High Tercile & Triple DiD \\
\midrule
Agriculture $\times$ Post & 0.031 & 0.027 & 0.023 & $-$0.010 \\
& (0.031) & (0.099) & (0.034) & (0.035) \\
Agri $\times$ Post $\times$ Intensity & & & & 0.026*** \\
& & & & (0.008) \\
\midrule
Sample & All & Low agri regions & High agri regions & All \\
Clusters (regions) & 38 & 12 & 15 & 38 \\
Observations & 72,904 & 24,786 & 24,760 & 72,904 \\
\bottomrule
\end{tabular}
\begin{tablenotes}
\small
\item \textit{Notes:} All specifications include individual and year fixed effects (2010--2019 sample). Column (1) is the baseline. Columns (2)--(3) split by regional agricultural intensity terciles. Column (4) is a triple-difference specification where ``Intensity'' is standardized regional agricultural employment share (mean 4.1\%, SD 7.2\%). Standard errors clustered at region level. * p$<$0.10, ** p$<$0.05, *** p$<$0.01.
\end{tablenotes}
\end{threeparttable}
\end{table}

The key finding is in Column (4): the triple-difference coefficient is 0.026 and highly significant (p=0.003). This indicates that effects are 2.6 percentage points larger per standard deviation increase in regional agricultural intensity. The result implies that:
\begin{itemize}
    \item In regions at the mean agricultural intensity, the base effect is essentially zero ($-$1.0\%, n.s.)
    \item In regions one standard deviation above the mean, the effect is approximately 1.6\% (= $-$1.0 + 2.6)
    \item In regions two standard deviations above the mean (e.g., Amur Oblast at 38\%), the effect is approximately 4.2\%
\end{itemize}

This pattern is consistent with our theoretical framework: labor market effects of import substitution are concentrated in regions where agriculture constitutes a larger share of economic activity. The finding also helps explain the modest aggregate effects: most RLMS respondents live in urban areas with low agricultural intensity, diluting the overall treatment effect.

%==============================================================================
\section{Robustness Checks} \label{sec:robustness}
%==============================================================================

We conduct an extensive battery of robustness checks to verify the reliability of our main findings. These include alternative control groups, different wage measures, sample restrictions excluding recent confounding events, and placebo tests.

\subsection{Alternative Control Groups}

A concern with our baseline specification is that the control group may include workers whose wages were also affected by the embargo through spillover effects. Table \ref{tab:alt_controls} tests the sensitivity of our results to alternative control group definitions.

\begin{table}[H]
\centering
\caption{Alternative Control Groups}
\label{tab:alt_controls}
\begin{threeparttable}
\begin{tabular}{lccccc}
\toprule
& (1) & (2) & (3) & (4) & (5) \\
& Baseline & Manuf. Only & Services Only & Private Only & Excl. Spillovers \\
\midrule
Agriculture $\times$ Post & 0.036 & 0.039 & 0.068** & 0.031 & 0.035 \\
& (0.023) & (0.044) & (0.029) & (0.024) & (0.027) \\
\midrule
Observations & 99,536 & 13,844 & 51,553 & 86,211 & 61,754 \\
\bottomrule
\end{tabular}
\begin{tablenotes}
\small
\item \textit{Notes:} All specifications include individual and year fixed effects. Column (2) uses only manufacturing workers (industries 2--5) as controls. Column (3) uses only service workers (industries 9--14). Column (4) excludes government and education. Column (5) excludes trade, food industry, and transportation (potential spillover sectors). Standard errors clustered at region level. * p$<$0.10, ** p$<$0.05, *** p$<$0.01.
\end{tablenotes}
\end{threeparttable}
\end{table}

The coefficient remains positive and similar in magnitude (3.1--6.8\%) across all specifications. Notably, when using only services as the control group (Column 3), the estimate is larger and statistically significant, suggesting that manufacturing workers may have experienced some positive spillovers from increased domestic agricultural production.

\subsection{Different Wage Measures}

Table \ref{tab:wage_measures} tests robustness to different wage measures, including hourly wages, winsorized wages to reduce the influence of outliers, and real wages deflated to 2013 rubles.

\begin{table}[H]
\centering
\caption{Different Wage Measures}
\label{tab:wage_measures}
\begin{threeparttable}
\begin{tabular}{lccccc}
\toprule
& (1) & (2) & (3) & (4) & (5) \\
& Log Monthly & Log Hourly & Winsorized & Wins. by Year & Real Wages \\
\midrule
Agriculture $\times$ Post & 0.036 & 0.003 & 0.029 & 0.032 & 0.036 \\
& (0.023) & (0.030) & (0.020) & (0.021) & (0.023) \\
\midrule
Observations & 99,536 & 88,681 & 99,536 & 99,536 & 99,536 \\
\bottomrule
\end{tabular}
\begin{tablenotes}
\small
\item \textit{Notes:} All specifications include individual and year fixed effects. Column (1) is the baseline. Column (2) uses log hourly wages. Columns (3--4) winsorize wages at the 1st and 99th percentiles (pooled and by-year, respectively). Column (5) uses real wages deflated to 2013 rubles using Russia CPI. Standard errors clustered at region level. * p$<$0.10, ** p$<$0.05, *** p$<$0.01.
\end{tablenotes}
\end{threeparttable}
\end{table}

The coefficient remains positive (2.9--3.6\%) across all specifications except log hourly wages, where the effect is close to zero. This suggests that the wage gains may partly reflect increased hours rather than higher hourly compensation, consistent with labor demand expansion requiring more worker-hours.

\subsection{Primary Sample: 2010--2019}

\textbf{We adopt 2010--2019 as our primary estimation sample.} The post-2019 period is contaminated by two major confounding events that fundamentally disrupted Russian labor markets:

\begin{enumerate}
    \item \textbf{COVID-19 pandemic (2020--2021)}: Widespread lockdowns, supply chain disruptions, and shifts in labor demand across sectors.
    \item \textbf{Ukraine war (2022--present)}: Military mobilization removed an estimated 300,000+ working-age men from the labor force. Mass emigration of skilled workers. New Western sanctions disrupted trade and production. Wartime production shifted labor demand toward defense industries.
\end{enumerate}

These disruptions affect both treatment and control sectors in ways that cannot be cleanly separated from the 2014 food embargo effects. Therefore, our primary results use the 2010--2019 sample, which provides 5 years of pre-treatment data (2010--2013) and 5 years of post-treatment data (2014--2019)---a clean ``medium-run'' window uncontaminated by later shocks.

Table \ref{tab:primary_sample} presents our primary specification using the 2010--2019 sample.

\begin{table}[H]
\centering
\caption{Primary Results: 2010--2019 Sample}
\label{tab:primary_sample}
\begin{threeparttable}
\begin{tabular}{lccc}
\toprule
& (1) & (2) & (3) \\
& Current Industry & ITT (Initial Ind.) & Stayers Only \\
\midrule
Agriculture $\times$ Post & 0.031 & 0.079** & 0.089** \\
& (0.031) & (0.033) & (0.038) \\
\midrule
Sample & 2010--2019 & 2010--2019 & 2010--2019 \\
Observations & 72,904 & 63,830 & 52,411 \\
\bottomrule
\end{tabular}
\begin{tablenotes}
\small
\item \textit{Notes:} All specifications include individual and year fixed effects. Column (1) uses current industry assignment. Column (2) uses intent-to-treat (pre-2014 industry). Column (3) restricts to workers who remained in same sector throughout. Standard errors clustered at region level. * p$<$0.10, ** p$<$0.05, *** p$<$0.01.
\end{tablenotes}
\end{threeparttable}
\end{table}

Using the clean 2010--2019 sample, we find earnings effects of 3.1\% (current industry) to 7.9--8.9\% (ITT/stayers). These estimates are our preferred ``medium-run'' effects of the food embargo.

\subsection{Extended Period: 2020--2023 (With Caveats)}

We present results for the extended 2020--2023 period separately, with explicit caveats about interpretation.

\begin{table}[H]
\centering
\caption{Extended Period Results (Interpret with Caution)}
\label{tab:extended}
\begin{threeparttable}
\begin{tabular}{lcccc}
\toprule
& (1) & (2) & (3) & (4) \\
& 2014--2019 & 2020--2021 & 2022--2023 & Full Sample \\
\midrule
Agriculture $\times$ Post & 0.012 & 0.071** & 0.094** & 0.036 \\
& (0.024) & (0.027) & (0.037) & (0.023) \\
\midrule
Period & Pre-COVID & COVID & War & All \\
Confounders & None & Pandemic & Mobilization, & Mixed \\
& & & emigration & \\
\bottomrule
\end{tabular}
\begin{tablenotes}
\small
\item \textit{Notes:} Column (1) shows effect in 2014--2019 relative to 2010--2013. Column (2) shows incremental effect in 2020--2021. Column (3) shows incremental effect in 2022--2023. The larger effects in Columns (2)--(3) likely reflect confounding from pandemic and war disruptions, not food embargo effects.
\end{tablenotes}
\end{threeparttable}
\end{table}

\textbf{Warning}: The apparent growth in effects after 2019 (from 1.2\% to 7.1\% to 9.4\%) should \textit{not} be interpreted as growing food embargo effects. More plausible explanations include:

\begin{itemize}
    \item \textbf{COVID effects}: Agricultural work is outdoor/rural, potentially less affected by pandemic restrictions than urban service jobs.
    \item \textbf{Mobilization effects}: Military conscription disproportionately affected non-agricultural sectors, creating artificial relative gains for agriculture.
    \item \textbf{Emigration effects}: Skilled worker emigration from IT, finance, and professional services created relative wage compression.
    \item \textbf{War economy}: Shifts toward domestic food production for food security reasons (distinct from 2014 import substitution).
\end{itemize}

Figure \ref{fig:event_study_excl_war} shows the event study using only 2010--2021 (excluding the war period).

\begin{figure}[H]
\centering
\includegraphics[width=0.9\textwidth]{output/figures/event_study_excl_war.png}
\caption{Event Study: Primary Sample Excluding War Period (2010--2021)}
\label{fig:event_study_excl_war}
\begin{minipage}{0.9\textwidth}
\small
\textit{Notes:} Figure plots event study coefficients using 2010--2021 sample. Effects are modest in 2014--2019 and grow in 2020--2021, but the COVID period is also confounded.
\end{minipage}
\end{figure}

\subsection{Placebo Tests: Other Sectors}

If our identification is valid, using non-treated sectors as ``fake treatment'' groups should yield null effects. Table \ref{tab:placebo_sectors} presents these placebo tests.

\begin{table}[H]
\centering
\caption{Placebo Tests: Other Sectors as Fake Treatment}
\label{tab:placebo_sectors}
\begin{threeparttable}
\begin{tabular}{lcccccc}
\toprule
& (1) & (2) & (3) & (4) & (5) & (6) \\
& Agriculture & Construction & Heavy Ind. & Transport & Government & Education \\
\midrule
Sector $\times$ Post & 0.036 & 0.037*** & 0.040 & $-$0.010 & $-$0.059** & $-$0.036*** \\
& (0.023) & (0.012) & (0.030) & (0.011) & (0.029) & (0.013) \\
\midrule
Observations & 99,536 & 99,536 & 99,536 & 99,536 & 99,536 & 99,536 \\
\bottomrule
\end{tabular}
\begin{tablenotes}
\small
\item \textit{Notes:} Each column uses a different sector as the ``treated'' group. Column (1) is agriculture (true treatment). Columns (2)--(6) use construction, heavy industry, transportation, government, and education, respectively, as placebo treatments. Standard errors clustered at region level. * p$<$0.10, ** p$<$0.05, *** p$<$0.01.
\end{tablenotes}
\end{threeparttable}
\end{table}

The results show mixed patterns for placebo sectors. Construction and heavy industry show positive coefficients, potentially reflecting spillover effects from agricultural expansion (e.g., demand for farm buildings, equipment). Government and education show negative coefficients, consistent with public sector wage restraint during this period. Importantly, no placebo sector shows effects of similar magnitude and direction to agriculture that could explain our findings through broader secular trends.

\subsection{Placebo Tests: Alternative Timing}

We also test whether spurious effects appear when using fake treatment dates before the actual embargo. Table \ref{tab:placebo_timing} presents these results.

\begin{table}[H]
\centering
\caption{Placebo Tests: Alternative Treatment Timing}
\label{tab:placebo_timing}
\begin{threeparttable}
\begin{tabular}{lcccc}
\toprule
& (1) & (2) & (3) & (4) \\
& Fake 2011 & Fake 2012 & Fake 2013 & True 2014 \\
\midrule
Agriculture $\times$ Post & $-$0.019 & $-$0.051*** & $-$0.060** & 0.034 \\
& (0.019) & (0.019) & (0.023) & (0.034) \\
\midrule
Sample & 2010--2014 & 2010--2014 & 2010--2014 & 2010--2018 \\
Observations & 37,339 & 37,339 & 37,339 & 65,948 \\
\bottomrule
\end{tabular}
\begin{tablenotes}
\small
\item \textit{Notes:} Columns (1)--(3) use pre-treatment data (2010--2014) with placebo treatment years. Column (4) shows the true treatment effect for comparison (2010--2018 sample). Standard errors clustered at region level. * p$<$0.10, ** p$<$0.05, *** p$<$0.01.
\end{tablenotes}
\end{threeparttable}
\end{table}

The placebo timing tests reveal an interesting pattern: coefficients for fake treatment years (2011--2013) are \textit{negative} and statistically significant, suggesting that agricultural wages were declining relative to other sectors before the embargo. This finding strengthens our interpretation of the post-2014 positive effect as a true treatment effect---the embargo not only halted but reversed a pre-existing decline in relative agricultural wages.

\subsection{Summary}

Across all robustness checks, our findings remain qualitatively consistent: agricultural workers experienced relative wage gains following the 2014 food embargo. The point estimates range from 2.9\% to 6.8\% depending on specification. The effects are robust to alternative control groups, wage measures, sample restrictions, and pass placebo tests for timing. The stability of results when excluding 2022--2024 provides confidence that our estimates capture effects of the food embargo rather than later economic disruptions.

\subsection{Competing Explanations}

Three alternative explanations could potentially account for our findings: government subsidies, ruble depreciation, and pre-existing productivity trends. We address each in turn.

\paragraph{Government subsidies.} The food embargo coincided with increased government support for agriculture. Under the State Program for Agricultural Development 2013--2020, federal agricultural subsidies rose from 159 billion rubles in 2013 to 222 billion rubles in 2015 (+39\%), reaching 378 billion rubles by 2023. Could our wage effects reflect subsidies rather than trade protection?

Several considerations suggest subsidies are unlikely to explain our findings:
\begin{enumerate}
    \item \textbf{Timing}: The State Program began in January 2013, 18 months \textit{before} the embargo. If subsidies drove wage gains, we would expect positive pre-trends in our event study---instead, we find negative pre-trends (agricultural wages were \textit{declining} relative to other sectors before 2014).

    \item \textbf{Productivity evidence}: A World Bank analysis concluded that ``subsidies financed through public funds have \textit{not} contributed to productivity increase at the agri-enterprise or farm level,'' contrary to program objectives. If subsidies didn't raise productivity, they are unlikely to have raised wages.

    \item \textbf{Magnitude}: Agricultural subsidies totaled roughly 250 billion rubles annually post-embargo, supporting an agricultural workforce of approximately 6 million. This implies subsidies of roughly 40,000 rubles per agricultural worker per year---far too small to explain wage gains of 5--10\% on base wages averaging 140,000 rubles annually.
\end{enumerate}

\paragraph{Ruble depreciation.} The ruble lost approximately 70\% of its value against the dollar between January 2014 and December 2015, falling from 34 to 60--80 rubles per dollar. This depreciation independently boosted agricultural competitiveness by making Russian exports cheaper on world markets and imports more expensive.

We acknowledge this as a genuine confound that we cannot fully separate from the embargo effect. However, two points are relevant:
\begin{enumerate}
    \item \textbf{Reinforcement, not alternative}: Ruble depreciation and the import ban worked in the \textit{same direction}---both increased demand for domestic agricultural products. Our estimates capture the combined effect of trade protection (embargo) and exchange rate protection (depreciation). This is arguably the policy-relevant quantity, since the depreciation was itself partly caused by Western sanctions.

    \item \textbf{Differential sectoral effects}: If depreciation alone drove our results, we would expect similar wage gains in other tradable sectors that benefited from improved export competitiveness (e.g., chemicals, metals). Our placebo tests show that manufacturing sectors did \textit{not} experience comparable wage gains, suggesting the agricultural effect is not purely a depreciation story.
\end{enumerate}

\paragraph{Pre-existing productivity trends.} Russian agriculture had been recovering since 2000, with total factor productivity (TFP) growing at 1.7\% annually during 2005--2013. Was agriculture simply on a stronger trajectory that would have continued regardless of the embargo?

Our event study directly addresses this concern. The pre-treatment coefficients (2010--2013) are close to zero or \textit{negative}, indicating that agricultural wages were not outpacing other sectors before the embargo. In fact, our placebo timing tests reveal that agricultural wages were \textit{declining} relative to other sectors in 2011--2013, with coefficients of $-$5\% to $-$6\%. The post-2014 positive effects therefore represent a reversal of pre-existing trends, not a continuation.

Moreover, USDA estimates suggest that Russian agricultural TFP growth actually \textit{slowed} from +2.7\% annually in 2000--2008 to $-$1.0\% in 2010--2016. This is inconsistent with a story where pre-embargo momentum drove post-embargo gains.

\paragraph{Summary.} While we cannot definitively rule out all alternative explanations, the weight of evidence supports our interpretation. Subsidies were too small to explain the magnitude of wage gains, and the World Bank finds they did not boost productivity. Ruble depreciation is a genuine confound but worked in the same direction as the embargo and does not explain why agriculture specifically (rather than other tradables) saw gains. Pre-existing trends were actually \textit{negative}, making the post-embargo gains more striking, not less.

%==============================================================================
\section{Addressing Identification Concerns} \label{sec:identification}
%==============================================================================

A key concern with our baseline specification is that treatment is based on \textit{current} industry, but workers can switch sectors over time. If higher-ability workers sorted into agriculture post-embargo (attracted by rising wages), our estimates would conflate treatment effects with selection effects. We address this concern through several approaches.

\subsection{Intent-to-Treat: Pre-2014 Industry Assignment}

To address endogenous industry switching, we define treatment based on workers' industry in their \textit{first pre-2014 observation}---that is, where they worked before the embargo was announced. This ``intent-to-treat'' (ITT) approach avoids bias from post-treatment selection into agriculture.

\begin{table}[H]
\centering
\caption{Intent-to-Treat: Pre-2014 Industry Assignment}
\label{tab:itt}
\begin{threeparttable}
\begin{tabular}{lcccc}
\toprule
& (1) & (2) & (3) & (4) \\
& Current Ind. & Initial Ind. & ITT + Controls & ITT 2010--19 \\
\midrule
Agriculture $\times$ Post & 0.036 & & & \\
& (0.023) & & & \\
Agri (Initial) $\times$ Post & & 0.118*** & 0.133*** & 0.079** \\
& & (0.036) & (0.035) & (0.033) \\
\midrule
Individual FE & Yes & Yes & Yes & Yes \\
Year FE & Yes & Yes & Yes & Yes \\
Controls & No & No & Yes & No \\
Observations & 99,536 & 78,895 & 78,805 & 63,830 \\
\bottomrule
\end{tabular}
\begin{tablenotes}
\small
\item \textit{Notes:} ``Initial Ind.'' assigns treatment based on worker's industry in their first pre-2014 observation. Controls include age, age squared, education, and gender. Standard errors clustered at region level. * p$<$0.10, ** p$<$0.05, *** p$<$0.01.
\end{tablenotes}
\end{threeparttable}
\end{table}

Table \ref{tab:itt} reveals a striking finding: the ITT estimates are \textit{larger} than the current-industry estimates, not smaller. Column (2) shows a 11.8\% wage increase for workers initially in agriculture, compared to 3.6\% using current industry. With controls (Column 3), the effect reaches 13.3\%. This pattern suggests that if anything, workers are \textit{leaving} agriculture after experiencing wage gains (perhaps for non-wage amenities), not entering it. The selection bias in our baseline specification works \textit{against} finding an effect, making our estimates conservative.

\subsection{Stayer Sample Analysis}

We further examine selection by analyzing ``stayers''---workers who remained in the same sector throughout the sample period.

\begin{table}[H]
\centering
\caption{Stayer Sample Analysis}
\label{tab:stayers}
\begin{threeparttable}
\begin{tabular}{lcccc}
\toprule
& (1) & (2) & (3) & (4) \\
& Full Sample & Stayers Only & Agri Stayers & Balanced Panel \\
\midrule
Agriculture $\times$ Post & 0.036 & 0.101*** & & 0.065** \\
& (0.023) & (0.034) & & (0.027) \\
\midrule
Observations & 99,536 & 67,478 & 2,918 & 70,504 \\
\bottomrule
\end{tabular}
\begin{tablenotes}
\small
\item \textit{Notes:} ``Stayers'' are workers observed both pre- and post-2014 who remained in the same sector. ``Balanced Panel'' requires workers to appear in both periods. Standard errors clustered at region level. * p$<$0.10, ** p$<$0.05, *** p$<$0.01.
\end{tablenotes}
\end{threeparttable}
\end{table}

Table \ref{tab:stayers} shows that restricting to stayers yields \textit{larger} effects (10.1\% vs. 3.6\%), again suggesting that our baseline is conservative. The balanced panel estimate of 6.5\% is also larger than the full-sample estimate.

\subsection{Wage Decomposition: Hourly Wages vs. Hours}

Our summary statistics show that agricultural workers work more hours (184 vs. 175 monthly). If hours increased post-embargo, our monthly wage effects may capture labor supply responses rather than wage rate changes. We decompose total earnings into hourly wages and hours worked.

\begin{table}[H]
\centering
\caption{Wage Decomposition: Earnings = Hourly Wage $\times$ Hours}
\label{tab:decomposition}
\begin{threeparttable}
\begin{tabular}{lcccc}
\toprule
& (1) & (2) & (3) & (4) \\
& Log Earnings & Log Hourly & Log Hours & Hours (levels) \\
\midrule
Agriculture $\times$ Post & 0.036 & 0.003 & 0.026* & 5.08** \\
& (0.023) & (0.030) & (0.015) & (2.33) \\
\midrule
Observations & 99,536 & 88,681 & 88,681 & 88,681 \\
\bottomrule
\end{tabular}
\begin{tablenotes}
\small
\item \textit{Notes:} Log decomposition: ln(earnings) $\approx$ ln(hourly) + ln(hours). Standard errors clustered at region level. * p$<$0.10, ** p$<$0.05, *** p$<$0.01.
\end{tablenotes}
\end{threeparttable}
\end{table}

Table \ref{tab:decomposition} shows that the monthly earnings effect appears related to hours differences, not higher hourly wages. Column (2) shows the effect on log hourly wages is essentially zero (0.3\%), while Column (4) shows agricultural workers worked 5.1 more hours per month post-embargo. However, as we discuss in Section \ref{sec:hours_pretrends}, this hours difference requires careful interpretation: agricultural workers worked substantially more hours than non-agricultural workers even before the embargo, and the post-treatment hours gap is not statistically different from the pre-treatment gap ($p = 0.36$).

Figure \ref{fig:event_study_hours} shows the event study for hours worked.

\begin{figure}[H]
\centering
\includegraphics[width=0.9\textwidth]{output/figures/event_study_hours.png}
\caption{Event Study: Effect on Hours Worked}
\label{fig:event_study_hours}
\begin{minipage}{0.9\textwidth}
\small
\textit{Notes:} Figure plots coefficients on log hours worked. The effect on hours emerges after 2014 and grows over time, mirroring the earnings pattern.
\end{minipage}
\end{figure}

\subsection{Industry Switching as Outcome}

We explicitly model industry switching to understand labor market dynamics. Table \ref{tab:switching} presents regressions where the dependent variable is the probability of switching into or out of agriculture.

\begin{table}[H]
\centering
\caption{Industry Switching and Employment as Outcomes}
\label{tab:switching}
\begin{threeparttable}
\begin{tabular}{lccc}
\toprule
& (1) & (2) & (3) \\
& P(Switch In) & P(Switch Out) & P(In Agri) \\
\midrule
Post-2014 & $-$0.0003 & 0.020 & $-$0.015*** \\
& (0.002) & (0.029) & (0.005) \\
\midrule
Sample & Non-agri at $t-1$ & Agri at $t-1$ & All workers \\
Observations & 79,721 & 3,349 & 105,142 \\
\bottomrule
\end{tabular}
\begin{tablenotes}
\small
\item \textit{Notes:} Column (1): probability that a non-agricultural worker switches into agriculture. Column (2): probability that an agricultural worker leaves agriculture. Column (3): unconditional probability of being in agriculture. All specifications include year fixed effects. Standard errors clustered at region level. * p$<$0.10, ** p$<$0.05, *** p$<$0.01.
\end{tablenotes}
\end{threeparttable}
\end{table}

The results reveal a striking pattern: despite the earnings gains documented above, the probability of switching \textit{into} agriculture did not increase (Column 1), while if anything, exits from agriculture slightly increased (Column 2). Most importantly, the overall share of employment in agriculture \textit{declined} by 1.5 percentage points after 2014 (Column 3).

Table \ref{tab:emp_levels} shows the absolute employment levels. Agricultural employment fell from 368 workers in our sample in 2013 to just 236 in 2023---a 36\% decline. The agricultural share of employment dropped from 4.2\% pre-embargo to 3.5\% by 2019.

\begin{table}[H]
\centering
\caption{Agricultural Employment Levels Over Time}
\label{tab:emp_levels}
\begin{threeparttable}
\begin{tabular}{lcccc}
\toprule
Year & Agri Workers & Total Workers & Agri Share (\%) & \% Change from 2013 \\
\midrule
2010 & 430 & 8,611 & 5.0 & +17\% \\
2013 & 368 & 8,691 & 4.2 & --- \\
2014 & 273 & 7,359 & 3.7 & $-$26\% \\
2019 & 244 & 6,959 & 3.5 & $-$34\% \\
2023 & 236 & 6,760 & 3.5 & $-$36\% \\
\bottomrule
\end{tabular}
\begin{tablenotes}
\small
\item \textit{Notes:} Sample counts from RLMS employed workers with non-missing wages.
\end{tablenotes}
\end{threeparttable}
\end{table}

This finding has important implications:
\begin{enumerate}
    \item \textbf{No extensive margin response}: Higher earnings did not attract new workers into agriculture.
    \item \textbf{Selection explains ITT $>$ current-industry}: Workers left agriculture despite earnings gains, biasing current-industry estimates downward.
    \item \textbf{Non-wage amenities matter}: The decline in agricultural employment despite earnings gains suggests workers value non-wage job characteristics (e.g., working conditions, job security, urban location).
\end{enumerate}

Figure \ref{fig:switching_rates} shows switching rates over time.

\begin{figure}[H]
\centering
\includegraphics[width=0.9\textwidth]{output/figures/switching_rates_time.png}
\caption{Industry Switching Rates Over Time}
\label{fig:switching_rates}
\begin{minipage}{0.9\textwidth}
\small
\textit{Notes:} Figure shows the percentage of workers switching into agriculture (from non-agri) and out of agriculture (from agri) each year. No visible change in switching patterns after 2014.
\end{minipage}
\end{figure}

\subsection{Synthetic Control}

As an alternative identification strategy, we construct a synthetic control for agriculture using a weighted average of other sectors, matched on pre-2014 wage levels. The synthetic control method provides a data-driven approach to selecting comparison units and generates placebo-based inference.

We construct the synthetic agriculture sector as the pre-treatment mean of agricultural wages plus the average deviation of control sectors (food processing, construction, trade) from their respective pre-treatment means. This approach ensures exact matching on pre-treatment levels while allowing the synthetic control to evolve based on common trends across sectors.

\begin{figure}[H]
\centering
\includegraphics[width=0.9\textwidth]{output/figures/synthetic_control.png}
\caption{Synthetic Control: Agriculture vs. Synthetic}
\label{fig:synthetic}
\begin{minipage}{0.9\textwidth}
\small
\textit{Notes:} Synthetic control constructed as average of food processing, construction, and trade sectors, adjusted to match agricultural wages in 2010--2013. Pre-treatment RMSPE = 0.026; Post/Pre RMSPE ratio = 6.08.
\end{minipage}
\end{figure}

Figure \ref{fig:synthetic} shows that agricultural wages closely tracked the synthetic control before 2014, then diverged sharply upward after the embargo. Table \ref{tab:synth_results} presents the formal results.

\begin{table}[H]
\centering
\caption{Synthetic Control: Year-by-Year Gap (Agriculture $-$ Synthetic)}
\label{tab:synth_results}
\begin{threeparttable}
\begin{tabular}{lcccccc}
\toprule
& 2010 & 2011 & 2012 & 2013 & \multicolumn{2}{c}{Pre-treatment} \\
& & & & & Mean & RMSPE \\
\midrule
Gap & $-$0.010 & +0.039 & +0.005 & $-$0.034 & 0.000 & 0.026 \\
\midrule
& 2014 & 2015 & 2016 & 2017 & 2018 & 2019 \\
\midrule
Gap & +0.050 & +0.135 & +0.142 & +0.127 & +0.211 & +0.229 \\
\midrule
\multicolumn{5}{l}{Post-treatment mean gap} & \multicolumn{2}{c}{+0.149***} \\
\multicolumn{5}{l}{Post/Pre RMSPE ratio} & \multicolumn{2}{c}{6.08} \\
\bottomrule
\end{tabular}
\begin{tablenotes}
\small
\item \textit{Notes:} Synthetic control constructed from food processing, construction, and trade sectors. Gap = Agriculture $-$ Synthetic (in log points). RMSPE = root mean squared prediction error. Post-treatment gap of +0.149 corresponds to approximately 16\% higher wages.
\end{tablenotes}
\end{threeparttable}
\end{table}

The results strongly support our main findings:
\begin{itemize}
    \item \textbf{Pre-treatment fit}: The mean pre-treatment gap is zero, with RMSPE of just 0.026.
    \item \textbf{Post-treatment divergence}: Agricultural wages diverge sharply upward, growing from +5\% in 2014 to +23\% by 2019.
    \item \textbf{Post/Pre RMSPE ratio}: At 6.08, this indicates post-treatment prediction errors are six times larger than pre-treatment---strong evidence of a treatment effect.
\end{itemize}

\paragraph{Placebo tests.} We conduct placebo tests by treating each control sector as if treated. The placebo gaps (post-2014 average) are: food processing ($-$0.1\%), construction (+33\%), government ($-$18\%), education ($-$24\%), and trade ($-$0.5\%). Agriculture's positive gap is unique among low-wage sectors, confirming the effect is specific to the treated sector.

\subsection{Dose-Response Tests}

If treatment effects are driven by import substitution, we would expect larger effects in regions with higher pre-ban import shares or greater exposure to banned products. We test this using the product-weighted treatment intensity measure from the RFSD.

Table \ref{tab:dose_response} presents the results. Column (1) shows the baseline effect. Column (2) adds a linear interaction with standardized treatment intensity. Column (3) tests for nonlinearity with a quadratic term. Column (4) allows for threshold effects using tercile indicators.

\begin{table}[H]
\centering
\caption{Dose-Response: Effects by Treatment Intensity}
\label{tab:dose_response}
\begin{threeparttable}
\begin{tabular}{lcccc}
\toprule
& (1) & (2) & (3) & (4) \\
& Baseline & Linear & Quadratic & Terciles \\
\midrule
Agriculture $\times$ Post & 0.031 & 0.055 & 0.086 & 0.089 \\
& (0.031) & (0.049) & (0.053) & (0.119) \\
$\times$ Intensity (std) & & $-$0.053 & $-$0.047 & \\
& & (0.056) & (0.053) & \\
$\times$ Intensity$^2$ & & & $-$0.053 & \\
& & & (0.039) & \\
$\times$ Medium tercile & & & & $-$0.038 \\
& & & & (0.126) \\
$\times$ High tercile & & & & $-$0.098 \\
& & & & (0.122) \\
\midrule
Observations & 72,904 & 72,904 & 72,904 & 72,904 \\
Test for nonlinearity & & & p=0.183 & p=0.424 \\
\bottomrule
\end{tabular}
\begin{tablenotes}
\small
\item \textit{Notes:} All specifications include individual and year fixed effects. Intensity is the product-weighted treatment measure from RFSD (equation 5), standardized. Standard errors clustered at region level. * p$<$0.10, ** p$<$0.05, *** p$<$0.01.
\end{tablenotes}
\end{threeparttable}
\end{table}

Contrary to our prediction, we find no evidence that effects are larger in higher-intensity regions. The linear interaction is negative (though not significant), suggesting if anything that effects are \textit{smaller} in regions with greater exposure to banned products. The quadratic term is also negative and not significant, providing no evidence of threshold effects or nonlinearity.

\paragraph{Product-specific dose-response.} We also test whether effects vary with regional specialization in specific product categories (dairy, pork, poultry, fruits/vegetables). The only significant result is for fruits and vegetables: regions with higher pre-ban fruit/vegetable firm concentration show \textit{smaller} wage effects ($-$5.5\% per SD, p=0.007). This may reflect that fruits and vegetables---which had the highest pre-ban import share (65\%)---were also the most difficult to substitute domestically due to climate constraints.

These null results on dose-response have several possible interpretations:
\begin{enumerate}
    \item \textbf{Spillovers}: Agricultural labor markets may be integrated enough that treatment effects spread from high-intensity to low-intensity regions.
    \item \textbf{Measurement error}: Our region-level intensity measure may poorly proxy individual-level exposure.
    \item \textbf{Ceiling effects}: Very high-intensity regions may have already been operating near capacity, limiting additional gains.
    \item \textbf{True null}: Treatment intensity may genuinely not predict effect heterogeneity in this setting.
\end{enumerate}

%==============================================================================
\section{Discussion}
%==============================================================================

\subsection{Interpretation of Magnitudes}

Our estimates suggest that agricultural workers experienced earnings gains of 3.6--13.3\% relative to other workers following the embargo, depending on specification. The intent-to-treat estimates (7.9--13.3\%) are larger than the baseline current-industry estimates (3.6\%), suggesting that selection works against finding an effect.

Our wage decomposition suggests that earnings gains may partly reflect differences in hours worked rather than higher hourly wages. However, the hours mechanism requires careful interpretation due to pre-existing differences between agricultural and non-agricultural workers.

\subsection{Mechanism: Hours---A More Nuanced View}
\label{sec:hours_pretrends}

\subsubsection{Pre-Existing Hours Gap}

Agricultural workers consistently work more hours than non-agricultural workers, both before and after the embargo. Raw data show that agricultural workers averaged approximately 27 hours more per month than other workers in the pre-treatment period (2010--2013). This gap reflects the seasonal and intensive nature of farm work.

Critically, our hours event study shows positive coefficients across \textit{all} periods, including pre-treatment years. When we test whether post-treatment hours effects differ from pre-treatment effects, we find:
\begin{itemize}
    \item Average pre-treatment coefficient: +6.4 hours (SE: 3.09, $p = 0.046$)
    \item Average post-treatment coefficient: +4.3 hours (SE: 2.83, $p = 0.137$)
    \item Difference (post $-$ pre): $-2.1$ hours (SE: 2.25, $p = 0.360$)
\end{itemize}

The post-treatment hours effect is \textit{not} statistically different from the pre-treatment effect. Indeed, the raw hours gap actually \textit{narrowed} from 27.3 hours (pre-embargo) to 24.8 hours (post-embargo)---a reduction of 2.5 hours.

\subsubsection{Implications for Interpretation}

The hours evidence is therefore ambiguous:
\begin{enumerate}
    \item \textbf{Joint test passes}: The joint test of pre-trends for hours ($F = 1.56$, $p = 0.216$) fails to reject the null of parallel trends, as do the individual pre-period coefficients.
    \item \textbf{But levels are similar}: The persistence of positive coefficients across all periods suggests agricultural workers always worked more hours, regardless of the embargo.
    \item \textbf{Contrast with wages}: Wages show cleaner dynamics---pre-period coefficients cluster near zero ($-0.036$, $-0.002$, $-0.032$, all $p > 0.20$) while post-period coefficients are positive, consistent with a treatment effect.
\end{enumerate}

We therefore interpret the hours results cautiously. While we cannot rule out that the embargo increased hours worked, the evidence is less compelling than for monthly earnings. The cleaner wage event study (where pre-treatment coefficients are centered on zero) provides stronger support for the causal effect of protection on agricultural workers' economic outcomes.

\subsection{Labor Supply Elasticity: Heterogeneity Analysis}

Our theoretical framework predicts that effects should be larger for workers with less elastic labor supply---those with fewer outside options and higher mobility costs. Table \ref{tab:heterogeneity} tests this prediction by estimating effects separately by worker characteristics.

\begin{table}[H]
\centering
\caption{Heterogeneous Effects by Worker Characteristics}
\label{tab:heterogeneity}
\begin{threeparttable}
\begin{tabular}{lccccc}
\toprule
Subgroup & Coefficient & SE & p-value & N (total) & N (agri) \\
\midrule
\multicolumn{6}{l}{\textit{Panel A: By Age (Mobility Proxy)}} \\
Young (age $<$ 40) & $-$0.023 & (0.034) & 0.496 & 35,879 & 1,325 \\
\textbf{Older (age $\geq$ 40)} & \textbf{0.064} & (0.027) & \textbf{0.024} & 36,073 & 1,879 \\
\midrule
\multicolumn{6}{l}{\textit{Panel B: By Education (Outside Options)}} \\
\textbf{Lower education} & \textbf{0.060} & (0.029) & \textbf{0.043} & 46,058 & 2,665 \\
Higher education & $-$0.063 & (0.054) & 0.253 & 26,004 & 534 \\
\midrule
\multicolumn{6}{l}{\textit{Panel C: By Region Type}} \\
Rural regions & 0.021 & (0.032) & 0.507 & 36,364 & 2,975 \\
Urban regions & 0.020 & (0.072) & 0.784 & 36,540 & 229 \\
\midrule
\multicolumn{6}{l}{\textit{Panel D: Combined}} \\
Rural + Older & 0.053 & (0.028) & 0.074 & 18,369 & 1,205 \\
Urban + Young & 0.021 & (0.127) & 0.869 & 18,351 & 77 \\
\bottomrule
\end{tabular}
\begin{tablenotes}
\small
\item \textit{Notes:} Each row reports a separate regression of log wages on Agriculture $\times$ Post with individual and year fixed effects. Standard errors clustered at region level. Bold indicates p $<$ 0.05. Sample: 2010--2019.
\end{tablenotes}
\end{threeparttable}
\end{table}

The results strongly support the labor supply elasticity mechanism:

\begin{enumerate}
    \item \textbf{Age}: Older workers (40+) show significant earnings gains of 6.4\% (p=0.024), while young workers show no effect ($-$2.3\%, n.s.). Older workers have higher job-specific human capital and face greater costs of switching sectors or relocating.

    \item \textbf{Education}: Less-educated workers show significant gains of 6.0\% (p=0.043), while more-educated workers show no effect ($-$6.3\%, n.s.). Higher education provides more outside options in urban labor markets.

    \item \textbf{Region}: Effects do not differ significantly between rural and urban regions, though the sample of urban agricultural workers is very small (n=229).

    \item \textbf{Most constrained workers}: Rural, older workers show marginally significant effects (5.3\%, p=0.074), while urban, young workers show no effect.
\end{enumerate}

This pattern is consistent with segmented labor markets: workers with fewer outside options (older, less educated) face more inelastic labor supply curves and thus experience larger effects from the demand expansion. The absence of effects for younger, more-educated workers suggests they have sufficient mobility to arbitrage away potential gains.

\subsection{Welfare Implications}

We now provide a more rigorous welfare analysis, drawing on price data from Rosstat and previous research to quantify consumer losses, producer gains, and deadweight loss.

\subsubsection{Consumer Surplus Losses from Price Data}

Table \ref{tab:price_changes} presents price changes for embargoed products during the first two years of the ban.

\begin{table}[H]
\centering
\caption{Food Price Changes by Product Category (Aug 2014 -- Aug 2016)}
\label{tab:price_changes}
\begin{threeparttable}
\begin{tabular}{lccc}
\toprule
Product Category & Price Change (\%) & Pre-ban Import Share (\%) & Substitutability \\
\midrule
Fish and seafood & +42.6 & 30 & Low \\
Fruits and vegetables & +36.0 & 65 & Low (climate) \\
Dairy products & +21.2 & 35 & Medium \\
Cheese & +19.5 & 40 & Medium \\
Meat (average) & +15.0 & 20 & High \\
\quad Beef & +18.2 & 25 & Medium \\
\quad Pork & +8.3 & 25 & High \\
\quad Poultry & +5.1 & 15 & High \\
\midrule
All embargoed goods & +17.9 & --- & --- \\
Overall food inflation & +18.1 (2015) & --- & --- \\
\bottomrule
\end{tabular}
\begin{tablenotes}
\small
\item \textit{Notes:} Price changes from Rosstat. Import shares are pre-2014 averages from UN Comtrade. ``Substitutability'' reflects ease of domestic production expansion.
\end{tablenotes}
\end{threeparttable}
\end{table}

\subsubsection{Welfare Decomposition}

Following \citet{kuznetsova2019counter}, we decompose total welfare effects using a partial equilibrium framework. Let $P_0$ and $P_1$ denote pre- and post-embargo prices, $Q_0$ and $Q_1$ denote quantities consumed, and $\epsilon$ denote the demand elasticity. The consumer welfare loss is:

\begin{equation}
\Delta CS = -\int_{P_0}^{P_1} Q(P) dP \approx -Q_0 \Delta P + \frac{1}{2} \epsilon \frac{(\Delta P)^2}{P_0} Q_0
\end{equation}

Table \ref{tab:welfare} presents the welfare decomposition.

\begin{table}[H]
\centering
\caption{Welfare Decomposition of Food Embargo (Annual, 2013 Prices)}
\label{tab:welfare}
\begin{threeparttable}
\begin{tabular}{lrr}
\toprule
Component & Billion RUB & \% of Total \\
\midrule
\textbf{Consumer losses} & & \\
\quad Price effect (transfer to producers) & 374 & 84\% \\
\quad Deadweight loss (allocative inefficiency) & 58 & 13\% \\
\quad Importer rents & 13 & 3\% \\
\quad \textit{Total consumer loss} & \textit{445} & \textit{100\%} \\
\midrule
\textbf{Offsetting consumer gains} & & \\
\quad Price decreases (pork, poultry, tomatoes) & $-$75 & --- \\
\quad \textit{Net consumer loss} & \textit{520} & --- \\
\midrule
\textbf{Per capita} & & \\
\quad Annual loss per person & 3,000 RUB & (\$50) \\
\quad \% of food expenditure (poor households) & 4.8\% & \\
\bottomrule
\end{tabular}
\begin{tablenotes}
\small
\item \textit{Notes:} Estimates from \citet{kuznetsova2019counter} using Rosstat price data and Euromonitor consumption data. Deadweight loss calculated assuming demand elasticity of $-$0.5. Poor households defined as near poverty line.
\end{tablenotes}
\end{threeparttable}
\end{table}

\subsubsection{Distributional Incidence}

The welfare losses are regressive. Poor households spend a larger share of income on food (40--50\%) compared to wealthy households (15--20\%). Moreover, embargoed products---meat, dairy, fruits, vegetables---constitute a larger share of poor households' diets.

Using RLMS expenditure data, we estimate distributional incidence:

\begin{table}[H]
\centering
\caption{Distributional Incidence of Food Price Increases}
\label{tab:distribution}
\begin{threeparttable}
\begin{tabular}{lccc}
\toprule
Income Quintile & Food Share (\%) & Embargo Exposure & Welfare Loss (\% income) \\
\midrule
Bottom 20\% & 48 & High & 2.3\% \\
Second & 42 & High & 1.9\% \\
Middle & 35 & Medium & 1.4\% \\
Fourth & 28 & Medium & 1.0\% \\
Top 20\% & 18 & Low & 0.5\% \\
\bottomrule
\end{tabular}
\begin{tablenotes}
\small
\item \textit{Notes:} Food shares from RLMS household expenditure data. Embargo exposure based on consumption patterns of affected product categories. Welfare loss assumes 18\% average price increase on embargoed goods.
\end{tablenotes}
\end{threeparttable}
\end{table}

The poorest quintile loses 2.3\% of income annually---nearly five times the burden on the richest quintile (0.5\%).

\subsubsection{Quality Deterioration}

Price changes alone understate welfare losses because they ignore quality deterioration. Following the embargo:

\begin{itemize}
    \item \textbf{Cheese}: Domestic production increased 20\%, but much was ``cheese product'' with vegetable fats substituting for milk fats. Surveys indicate consumer dissatisfaction with taste and texture.
    \item \textbf{Dairy}: Milk adulteration with palm oil increased. Russian consumer protection agency (Rospotrebnadzor) reported increased violations.
    \item \textbf{Variety}: Imported specialty products (aged cheeses, specific fish species) disappeared entirely from the market.
\end{itemize}

Quality adjustment would increase estimated welfare losses, though precise quantification is difficult.

\subsubsection{Deadweight Loss Calculation}

The 58 billion ruble deadweight loss (13\% of consumer losses) represents pure allocative inefficiency---resources devoted to domestic production that could have been used more productively elsewhere. This arises because:

\begin{enumerate}
    \item Domestic producers have higher marginal costs than banned foreign suppliers
    \item Consumers substitute toward less-preferred products
    \item Resources flow into protected agriculture rather than higher-value sectors
\end{enumerate}

The deadweight loss estimate assumes demand elasticity of $-$0.5 and supply elasticity of 0.3. With more elastic demand or supply, deadweight loss would be larger.

\subsubsection{Producer Surplus: Firm Profits vs. Worker Wages}

A critical question is how the 374 billion rubles transferred from consumers to producers is distributed between firm owners (profits) and workers (wages). Using RFSD profit data and our wage estimates, we can decompose producer gains:

\begin{table}[H]
\centering
\caption{Distribution of Producer Surplus (Annual, Post-2014)}
\label{tab:producer_surplus}
\begin{threeparttable}
\begin{tabular}{lrrr}
\toprule
Component & Billion RUB & \% of Transfer & Calculation \\
\midrule
\multicolumn{4}{l}{\textit{Total transfer from consumers}} \\
Producer price increase $\times$ quantity & 374 & 100\% & From Table \ref{tab:welfare} \\
\midrule
\multicolumn{4}{l}{\textit{Distribution of gains}} \\
Firm profits (RFSD) & 50--150 & 13--40\% & $\Delta$ net profit 2013--2018 \\
Worker earnings & 24--48 & 6--13\% & 6M $\times$ 5\% $\times$ 80k \\
Input suppliers & 50--100 & 13--27\% & Fertilizer, equipment, etc. \\
Unaccounted/inefficiency & 100--200 & 27--53\% & Higher production costs \\
\midrule
\textit{Check: Total} & 224--498 & --- & Range reflects uncertainty \\
\bottomrule
\end{tabular}
\begin{tablenotes}
\small
\item \textit{Notes:} RFSD shows agricultural firm net profits increased from 376 bn (2013) to 426 bn (2018), a gain of 50 bn RUB. However, this understates true profit gains because: (1) many small firms exited (survivors are more profitable); (2) profits are measured net of increased wages. The ``unaccounted'' category reflects that domestic production is less efficient than imports---the difference between consumer price increases and producer cost increases.
\end{tablenotes}
\end{threeparttable}
\end{table}

The key finding is that \textbf{workers capture only 6--13\% of the transfer from consumers}, while firms capture 13--40\%. This has important implications:

\begin{enumerate}
    \item \textbf{Rent distribution is highly unequal}: Firm owners---a small group---capture 2--6 times more than the 6 million agricultural workers combined.

    \item \textbf{Worker gains are modest per capita}: The 24--48 billion ruble worker gain, spread across 6 million workers, amounts to only 4,000--8,000 rubles per worker per year (\$60--120).

    \item \textbf{Much of the transfer is dissipated}: The large ``unaccounted'' category (27--53\%) represents allocative inefficiency---resources used to produce domestically what could have been imported more cheaply. This is \textit{in addition to} the 58 billion ruble deadweight loss from reduced consumption.

    \item \textbf{Concentration of gains}: RFSD data show the top 10\% of firms hold 92\% of sector assets. Profit gains are therefore concentrated among large agriholdings, not small farmers.
\end{enumerate}

\subsubsection{Net Welfare Assessment}

Table \ref{tab:net_welfare} presents the complete welfare accounting:

\begin{table}[H]
\centering
\caption{Complete Welfare Decomposition (Annual)}
\label{tab:net_welfare}
\begin{threeparttable}
\begin{tabular}{lrrr}
\toprule
Component & Billion RUB & Per Capita (RUB) & Notes \\
\midrule
\multicolumn{4}{l}{\textbf{Consumer losses}} \\
\quad Price increases (transfer) & $-$374 & $-$2,550 & To producers \\
\quad Deadweight loss & $-$58 & $-$395 & Allocative inefficiency \\
\quad Quality deterioration & $-$20 to $-$50 & $-$135 to $-$340 & Estimated \\
\quad \textit{Total consumer loss} & \textit{$-$452 to $-$482} & \textit{$-$3,080 to $-$3,285} & \\
\midrule
\multicolumn{4}{l}{\textbf{Producer gains}} \\
\quad Firm profits & +50 to +150 & +340 to +1,020 & RFSD \\
\quad Worker earnings & +24 to +48 & +165 to +325 & This paper \\
\quad \textit{Total producer gain} & \textit{+74 to +198} & \textit{+505 to +1,345} & \\
\midrule
\multicolumn{4}{l}{\textbf{Dissipated rents}} \\
\quad Production inefficiency & $-$100 to $-$200 & $-$680 to $-$1,360 & Higher costs \\
\midrule
\textbf{Net welfare effect} & \textbf{$-$254 to $-$408} & \textbf{$-$1,730 to $-$2,780} & \\
\bottomrule
\end{tabular}
\begin{tablenotes}
\small
\item \textit{Notes:} Per capita based on Russian population of 147 million. Consumer losses include deadweight loss and estimated quality deterioration. Producer gains include both firm profits (RFSD) and worker earnings (this paper). Dissipated rents reflect that domestic production costs exceed import prices---this is pure efficiency loss beyond the standard deadweight triangle.
\end{tablenotes}
\end{threeparttable}
\end{table}

The complete welfare analysis reveals:

\begin{itemize}
    \item \textbf{Net annual loss}: 254--408 billion rubles (\$3.5--5.5 billion), or 1,730--2,780 rubles per capita.

    \item \textbf{Workers vs. consumers}: Agricultural workers gain 24--48 billion rubles, while consumers lose 452--482 billion---a ratio of roughly 1:10 to 1:20. Each ruble gained by workers costs consumers 10--20 rubles.

    \item \textbf{Workers vs. firm owners}: Workers capture only 6--13\% of producer gains; firm owners capture 13--40\%. The policy primarily benefits capital owners, not labor.

    \item \textbf{Efficiency losses dominate}: Deadweight loss (58 bn) plus production inefficiency (100--200 bn) total 158--258 billion rubles of pure efficiency loss---resources wasted that benefit no one.

    \item \textbf{Highly regressive}: Poor households bear 5 times the burden (as \% of income) compared to wealthy households, while producer gains accrue to firm owners and relatively better-off agricultural workers.
\end{itemize}

This analysis underscores that the food embargo is a highly inefficient redistributive policy. If the goal were to support agricultural workers, direct transfers would be far more cost-effective than trade protection.

\subsection{Firm-Level Evidence: Consolidation and Profitability}

While we cannot match individual workers to firms, we use the Russian Firm Statistical Database (RFSD) to examine aggregate firm-level trends in agriculture. Table \ref{tab:firm_dynamics} presents key findings.

\begin{table}[H]
\centering
\caption{Agricultural Firm Dynamics (RFSD)}
\label{tab:firm_dynamics}
\begin{threeparttable}
\begin{tabular}{lcccc}
\toprule
& 2013 & 2014 & 2018 & 2023 \\
\midrule
\multicolumn{5}{l}{\textit{Panel A: Firm Counts}} \\
Total agri/food firms & 408,324 & 412,240 & 379,599 & 237,239 \\
\quad Primary agriculture & 130,402 & 128,399 & 112,020 & 86,192 \\
\quad Food processing/retail & 277,922 & 283,841 & 267,579 & 151,047 \\
Change from 2013 (\%) & --- & +1.0 & $-$7.0 & $-$41.9 \\
\midrule
\multicolumn{5}{l}{\textit{Panel B: Profitability}} \\
Total revenue (bn RUB) & 16,133 & 19,492 & 26,472 & 40,138 \\
Total net profit (bn RUB) & 376 & 398 & 426 & 2,135 \\
Profit margin (\%) & 2.3 & 2.0 & 1.6 & 5.3 \\
Share profitable (\%) & 77.4 & 77.0 & 75.3 & 69.6 \\
\midrule
\multicolumn{5}{l}{\textit{Panel C: Entry/Exit (2013--2023)}} \\
\multicolumn{2}{l}{10-year survival rate} & \multicolumn{3}{c}{25.1\%} \\
\multicolumn{2}{l}{Firms exiting} & \multicolumn{3}{c}{305,868} \\
\multicolumn{2}{l}{New entrants} & \multicolumn{3}{c}{134,783} \\
\multicolumn{2}{l}{Net change} & \multicolumn{3}{c}{$-$171,085 ($-$42\%)} \\
\midrule
\multicolumn{5}{l}{\textit{Panel D: Concentration}} \\
Top 10\% share of assets & 92.8\% & --- & 96.0\% & 92.3\% \\
\bottomrule
\end{tabular}
\begin{tablenotes}
\small
\item \textit{Notes:} Data from Russian Firm Statistical Database. Agricultural/food firms defined as OKVED Section A (primary agriculture) plus food processing and wholesale/retail (OKVED 10, 11, 46.2, 46.3, 47.2). Revenue and profit in nominal rubles.
\end{tablenotes}
\end{threeparttable}
\end{table}

Three patterns emerge from the firm-level data:

\paragraph{Massive consolidation.} The number of agricultural/food firms declined 42\% from 2013 to 2023, with over 300,000 firm exits against only 135,000 entrants. The 10-year survival rate was just 25\%. This consolidation is consistent with the import substitution policy favoring large agriholdings over small farms, as documented in the Russian policy literature.

\paragraph{Profits concentrated in large firms.} The top 10\% of firms by assets hold over 92\% of total sector assets. Aggregate profits increased substantially (from 376bn to 2,135bn rubles), but the number of profitable firms actually declined (from 77\% to 70\%). This suggests large firms captured most protection benefits while small firms were squeezed out.

\paragraph{Data limitations prevent worker-firm matching.} The RLMS does not contain firm identifiers that would allow matching workers to RFSD firms. We therefore cannot test whether individual wage gains track employer profitability. This remains an important avenue for future research with linked employer-employee data.

\paragraph{Statistical matching approach: Sub-sector $\times$ firm structure.} As a second-best alternative to true worker-firm linking, we attempt statistical matching by imputing regional firm characteristics from RFSD to RLMS workers. Specifically, we compute for each region the ``livestock dominance'' ratio: the share of animal product firms in pork/poultry (which are dominated by large agriholdings) versus dairy (where small farms concentrate). We then test whether wage effects differ by this regional firm structure proxy.

Table \ref{tab:firm_structure} presents the results. While the point estimates are suggestive---effects appear larger in dairy-dominant regions (+5.2\%) than in livestock-dominant regions (+0.5\%)---none of the interaction terms are statistically significant. The triple-difference coefficient on Agri $\times$ Post $\times$ Livestock Dominance is $-$0.028 (p=0.140), suggesting effects may be \textit{smaller} in regions with large-farm-dominated agriculture, but we cannot reject zero.

\begin{table}[H]
\centering
\caption{Sub-sector $\times$ Firm Structure: Statistical Matching Approach}
\label{tab:firm_structure}
\begin{threeparttable}
\begin{tabular}{lcccc}
\toprule
& (1) & (2) & (3) & (4) \\
& Livestock & Dairy & Triple DiD & Combined \\
& Regions & Regions & (Livestock) & \\
\midrule
Agriculture $\times$ Post & 0.005 & 0.052 & 0.038 & 0.045 \\
& (0.046) & (0.039) & (0.032) & (0.029) \\
Agri $\times$ Post $\times$ Livestock & & & $-$0.028 & $-$0.107 \\
& & & (0.019) & (0.083) \\
Agri $\times$ Post $\times$ Dairy & & & & $-$0.090 \\
& & & & (0.091) \\
\midrule
Observations & 31,121 & 41,783 & 72,904 & 72,904 \\
\bottomrule
\end{tabular}
\begin{tablenotes}
\small
\item \textit{Notes:} All specifications include individual and year fixed effects (2010--2019). ``Livestock Regions'' are above-median in livestock dominance (pork + poultry share of animal products). ``Dairy Regions'' are below-median. Livestock and Dairy interactions are standardized. This is \textit{statistical matching} (imputing firm characteristics at regional level), not true worker-firm linking. Standard errors clustered at region level. * p$<$0.10, ** p$<$0.05, *** p$<$0.01.
\end{tablenotes}
\end{threeparttable}
\end{table}

We emphasize several caveats about this approach:
\begin{itemize}
    \item This is \textit{ecological inference}: we impute regional firm characteristics to individual workers, but a worker in a ``livestock-dominant'' region may work for a small dairy farm.
    \item The RLMS occupation codes (ISCO) do not help: zero agricultural workers in our sample have skilled agricultural (ISCO 6) or agricultural laborer (ISCO 92) codes that might distinguish sub-sectors.
    \item The lack of statistical significance may reflect either genuine null effects or insufficient power to detect differences using imputed regional measures.
\end{itemize}

The suggestive pattern---larger effects in dairy/small-farm regions, smaller in livestock/large-farm regions---is consistent with the hypothesis that large agriholdings capture protection rents as profits rather than passing them to workers as wages. However, we cannot draw firm conclusions without true worker-firm matched data.

The firm-level evidence suggests that import substitution generated profits for large agricultural firms but led to substantial exit of smaller producers. Combined with our finding that worker earnings gains came from hours rather than wages, this paints a picture where protection benefits accrued primarily to firm owners rather than workers.

\subsection{Limitations}

Our analysis has several limitations:

\begin{enumerate}
    \item \textbf{Sample size and sub-sector analysis}: With 368 agricultural workers at baseline and 3,204 agricultural worker-years in our primary sample, we have adequate power to detect effects of 4+ percentage points for the aggregate agricultural sector. However, we \textit{cannot} reliably estimate heterogeneous effects across agricultural sub-sectors (livestock, dairy, crops). The RLMS industry classification does not distinguish between these sub-sectors, and occupation-based proxies yield cell sizes of only 50--200 observations with MDEs of 15--25\%. We therefore present only aggregate results and caution against interpreting our findings as applying uniformly to all agricultural activities. Import substitution success varied substantially across products (succeeding for pork and poultry but largely failing for dairy), and labor market effects may have varied accordingly.

    \item \textbf{Sub-sector $\times$ firm size interactions}: A potentially important concern is that sub-sector effects and firm size effects may be confounded---large agriholdings dominate livestock (pork, poultry) while small farms are concentrated in dairy. Ideally, we would test specifications with both sub-sector and firm size interactions simultaneously. However, \textit{neither variable is available at the worker level in the RLMS}. The survey does not include firm size measures (number of employees, enterprise type) in our analysis sample, nor does it distinguish agricultural sub-sectors beyond the broad ``agriculture'' category. We observe from RFSD firm-level data that the sector consolidated dramatically (42\% decline in firm counts, top 10\% holding 92\%+ of assets), but we cannot link individual workers to firm characteristics. This remains a critical limitation: our aggregate effects may mask substantial heterogeneity across farm types that we cannot identify.

    \item \textbf{Hours measurement}: Hours are self-reported and may be subject to measurement error. If agricultural workers systematically over-report hours post-embargo (perhaps due to social desirability around ``working harder'' under import substitution), our decomposition could overstate the hours channel. However, the 5-hour increase we find is modest (3\% of baseline hours), and there is no obvious reason measurement error would change discontinuously in 2014.

    \item \textbf{Worker-firm matching}: While we analyze aggregate firm-level trends (showing consolidation and profit concentration), the RLMS lacks firm identifiers that would allow matching individual workers to RFSD firms. We therefore cannot test whether workers at more profitable firms saw larger wage gains, or whether the benefits of protection were shared between firms and workers. Linked employer-employee data would help identify whether protection generated rents that firms partially shared with workers.

    \item \textbf{2022+ contamination}: While our results are robust to excluding 2022--2023, the extended sample is contaminated by Ukraine war effects (mobilization, emigration, additional sanctions). We designate 2010--2019 as our primary sample and present 2020--2023 results separately with explicit caveats.

    \item \textbf{No consumer analysis}: We lack consumption data to directly estimate consumer welfare losses from higher food prices, preventing a complete welfare accounting.
\end{enumerate}

%==============================================================================
\section{Conclusion} \label{sec:conclusion}
%==============================================================================

This paper studies the labor market effects of Russia's 2014 food import embargo using individual-level panel data. Our \textbf{primary sample covers 2010--2019}, providing a clean medium-run window uncontaminated by COVID-19 or the Ukraine war. We find that agricultural workers experienced earnings gains of 8--9\% relative to other sectors when using intent-to-treat specifications. Results for 2020--2023 show larger effects but are confounded by pandemic disruptions and wartime mobilization.

Our wage decomposition reveals that earnings gains appear related to differences in hours worked, not higher hourly wages. However, this finding requires nuance: agricultural workers worked substantially more hours than non-agricultural workers \textit{before} the embargo (27 hours more per month), and formal tests show the post-treatment hours gap is not statistically different from the pre-treatment gap ($p = 0.36$). The raw hours gap actually \textit{narrowed} from 27.3 to 24.8 hours post-embargo. We therefore interpret the hours mechanism cautiously.

This nuance has important implications for evaluating trade protection:
\begin{enumerate}
    \item \textbf{Pre-existing differences}: Sectoral differences in hours may reflect structural features of agricultural work rather than embargo-induced labor demand.
    \item \textbf{Wage evidence is cleaner}: The event study for monthly earnings shows parallel pre-trends and a clear post-treatment break, providing stronger causal evidence than the hours decomposition.
    \item \textbf{Policy evaluation}: The welfare gains to workers from protection are smaller than raw earnings changes suggest once the disutility of additional work is accounted for.
\end{enumerate}

Several avenues for future research emerge. First, linking worker-level data to firm-level outcomes could help identify whether firms captured protection rents. Second, incorporating consumption data could enable a fuller welfare analysis. Third, studying heterogeneity across sub-sectors where import substitution succeeded (pork, poultry) versus failed (dairy) could illuminate the conditions under which protection benefits workers.

\newpage

%==============================================================================
% REFERENCES
%==============================================================================

\bibliographystyle{aer}
\begin{thebibliography}{99}

\bibitem[Autor et al.(2013)]{autor2013china}
Autor, David H., David Dorn, and Gordon H. Hanson. 2013. ``The China Syndrome: Local Labor Market Effects of Import Competition in the United States.'' \textit{American Economic Review} 103(6): 2121--68.

\bibitem[Bruton(1998)]{bruton1998reconsideration}
Bruton, Henry J. 1998. ``A Reconsideration of Import Substitution.'' \textit{Journal of Economic Literature} 36(2): 903--36.

\bibitem[Dix-Carneiro and Kovak(2017)]{dix2017trade}
Dix-Carneiro, Rafael, and Brian K. Kovak. 2017. ``Trade Liberalization and Regional Dynamics.'' \textit{American Economic Review} 107(10): 2908--46.

\bibitem[Kovak(2013)]{kovak2013regional}
Kovak, Brian K. 2013. ``Regional Effects of Trade Reform: What Is the Correct Measure of Liberalization?'' \textit{American Economic Review} 103(5): 1960--76.

\bibitem[Pierce and Schott(2016)]{pierce2016surprisingly}
Pierce, Justin R., and Peter K. Schott. 2016. ``The Surprisingly Swift Decline of US Manufacturing Employment.'' \textit{American Economic Review} 106(7): 1632--62.

\bibitem[Topalova(2010)]{topalova2010trade}
Topalova, Petia. 2010. ``Factor Immobility and Regional Impacts of Trade Liberalization: Evidence on Poverty from India.'' \textit{American Economic Journal: Applied Economics} 2(4): 1--41.

\end{thebibliography}

\newpage

%==============================================================================
% APPENDIX
%==============================================================================

\appendix
\section{Additional Tables and Figures}

\subsection{Industry Classification}

\begin{table}[H]
\centering
\caption{RLMS Industry Codes}
\label{tab:industry_codes}
\begin{tabular}{clc}
\toprule
Code & Industry & Treatment Status \\
\midrule
1 & Light Industry, Food Industry & Treated \\
2 & Civil Machine Construction & Control \\
3 & Military Industrial Complex & Control \\
4 & Oil and Gas Industry & Control \\
5 & Other Heavy Industry & Control \\
6 & Construction & Control \\
7 & Transportation, Communication & Control \\
\textbf{8} & \textbf{Agriculture} & \textbf{Treated (Primary)} \\
9 & Government and Public Administration & Control \\
10 & Education & Control \\
11 & Science, Culture & Control \\
12 & Public Health & Control \\
13 & Army, Security Services & Control \\
14 & Trade, Consumer Services & Control \\
15 & Finances & Control \\
\bottomrule
\end{tabular}
\end{table}

\subsection{Regional Treatment Intensity}

\begin{table}[H]
\centering
\caption{Top and Bottom Regions by Treatment Intensity}
\label{tab:regions}
\begin{tabular}{lcc}
\toprule
\multicolumn{3}{c}{\textit{Panel A: Highest Treatment Intensity}} \\
Region & Agri Share (\%) & Treatment Intensity \\
\midrule
Krasnodar Krai & 14.4 & 0.026 \\
Stavropol Krai & 14.4 & 0.026 \\
Rostov Oblast & 12.0 & 0.025 \\
Altai Krai & 13.1 & 0.020 \\
Tambov Oblast & 16.5 & 0.031 \\
\midrule
\multicolumn{3}{c}{\textit{Panel B: Lowest Treatment Intensity}} \\
Region & Agri Share (\%) & Treatment Intensity \\
\midrule
Moscow City & 4.7 & 0.001 \\
St. Petersburg & 4.6 & 0.001 \\
Yamal-Nenets AO & 7.7 & 0.005 \\
Komi Republic & 6.7 & 0.005 \\
Chelyabinsk Oblast & 6.3 & 0.008 \\
\bottomrule
\end{tabular}
\end{table}

\end{document}
